\documentclass{article}

\usepackage{program}
\usepackage{latexsym}

\title{Symmetric Version of DUST Algorithm}
\author{Aleksandr Morgulis}

\newcommand{\al}[1]{\alpha_{#1}}
\newcommand{\sq}[2]{\al{#1}\ldots\al{#2}}
\newcommand{\cala}{{\cal A}}
\newcommand{\cals}{{\cal S}}

\newcommand{\OF}{\ \keyword{of}\ }
\newcommand{\RET}{\keyword{return}\ }
\newcommand{\VARP}{\keyword{var}\ }
\newcommand{\LIST}{\keyword{list}\ }
\newcommand{\TUPLE}{\keyword{tuple}}
\newcommand{\NIL}{\ \keyword{nil}\ }
\newcommand{\TYPE}{\keyword{type}\ }
\newcommand{\DODO}{\keyword{do}\tab}
\newcommand{\DOWHILE}{\untab\keyword{while}\ }

\newtheorem{lemma}{Lemma}

\setlength{\parskip}{1ex}
\setlength{\parindent}{0mm}

\begin{document}

\maketitle
\tableofcontents

\section{Definitions and Notation}

Let ${\cal S} = \sq{0}{N-1}$  be a sequence of $N$ letters from $4$-letter 
alphabet $\cala = \{ \mathbf A, \mathbf C, \mathbf G, \mathbf T \}$.

A {\em triplet} is a sequence of length $3$. Any sequence $\cals = \sq{0}{N-1}$ 
of length $N > 2$ contains as subsequences exactly $N-2$ triplets, that are 
denoted $t_{0},\ldots,t_{N-3}$, where $t_i = \al{i}\al{i+1}\al{i+2}$.

We define a mapping $d$ from $\cala$ to the subset of integers 
$\{0, 1, 2, 3\}$ in the following way: $d(\mathbf A) = 0$; $d(\mathbf C) = 1$; 
$d(\mathbf G) = 2$; $d(\mathbf T) = 3$.

To any triplet $t = \al{i}\al{i+1}\al{i+2}$ we assign an integer 
{\em triplet value} $v(t) = 16d(\al{i}) + 4d(\al{i+1}) + d(\al{i+2})$. This 
defines a one-to-one mapping from $64$ possible 3-letter sequences to the subset 
$\{0..63\}$ of integers.  In this text the triplets and their values are used 
interchangeably.

We assign a {\em score} $S(\cals)$ to any sequence $\cals$ of length $N > 3$ in 
the following way. Let $k_t :\ 0 \le t \le 63$ be the number of times a triplet 
with value $t$ appears in $\cals$. We define
$$
S(\cals) = \frac{\sum_{t=0}^{63}{k_t(k_t - 1)}}{2(l - 1)} \rm{\ ,}
$$
where $l = N - 2$ is the number of triplets in $\cals$. In other words, each
triplet value $t$ contributes $1 + 2 + \cdots + (k_t - 1)$ to the score.

Given an input nucleotide sequence $S$, window length $W$, and score threshold
$T$, let $|LC|(S)$ denote a set of subsequences of $S$ of length at most $W$ with
score greater than $T$. Both the original and the new DUST algorithms mask a subset 
of $|LC|(S)$. The bases masked by the new algorithm form a superset of those masked 
by the original one.

The algorithms below are presented in Pascal-like pseudocode. All undeclared
functions and types are described in section \ref{secpseudo}. Keyword \VARP
is used to declare local variables in procedure/function. When used in
the procedure/function parameter list it means that the corresponding parameter
is passed by reference, rather than by value.

\section{Original Algorithm}

The original DUST algorithm looks at all subsequences, called {\em windows},
of fixed length $W$ (subsequences of length less than $W$ are considered at 
the beginning and at the end of the input sequence) of the input nucleotide 
sequence.  For each such subsequence $\cals$ it finds the high scoring prefix 
$\cals'$ with the largest score (ties are resolved in favor of the leftmost
such prefix). If the score of the selected prefix is greater than a given
threshold value $T$ then the algorithm finds a subsequence $\cals''$ of $\cals'$ 
with the maximum score (again the ties are resolved in favor of the leftmost 
subsequence). The algorithm then masks the union of all such $\cals''$ over 
all the windows in the input sequence.

Below is the original DUST algorithm in pseudocode.

\NumberProgramstrue
\begin{program}
\BEGIN
\TYPE |prefix_info| := \TUPLE ( |end| : |int|,\ |score| : |real| );
\COMMENT{ The first argument to |best\_prefix|() is used for both input }
\COMMENT{ and output. The caller passes the input sequence, which }
\COMMENT{ is modified by the function to contain its highest scoring }
\COMMENT{ prefix. The actual score of the highest scoring prefix is }
\COMMENT{ the return value of the function. }
\FUNCT |best_prefix|( |Seq|:|sequence|) : |prefix_info| \BODY
    \BEGIN
    \VARP i, t, |i_max|, |sum| : |int|;
    \VARP |result| : |prefix_info|;
    \VARP |max| : |real|;
    \VARP |counts| : \ARRAY [0..63] \OF |int|;
    \FOR i:=0 \TO 63 \DO |counts|[i] := 0 \OD;
    |i_max| := 0;
    |max|   := 0;
    |sum|   := 0;
    \FOR i:=0 \TO |length|(|Seq|) - 3 \DO
        t           := |triplet|(|Seq|, i);
        |sum|       := |sum| + |counts|[t];
        |counts|[t] := |counts|[t] + 1;
        \IF i>0 \AND |sum|/i > |max| 
        \THEN |max|   := |sum|/i;
              |i_max| := i;
        \FI
    \OD;
    |Seq| := |subsequence|(|Seq|, 0, |i_max|);
    |result|.|end|   := |i_max|;
    |result|.|score| := |max|;
    \RET |result|;
    \END
\ENDFUNCT
\PROC |dust|( |Seq| : |sequence|,\ |Tres| : |real|,\ |Wsize| : |int|,
             \VARP |Res| : |list_of_integer_pairs| )\BODY
    \COMMENT{ Results are returned in |Res|. }
    \BEGIN
    \VARP i, j, a, b, |start|, |end|, |lim| : |int|;
    \VARP |score|, |suffix_score| : |real|;
    \VARP |window|, |suffix| : |sequence|;
    \VARP |winprefix|, |prefix| : |prefix_info|;
    \FOR i:=3 \TO |length|(|Seq|) + |Wsize| - 4 \DO
        \COMMENT{In the first and last $|Wsize| - 4$ iterations}
        \COMMENT{windows that are shorter than |Wsize| bases}
        \COMMENT{long are considered. Those are the windows}
        \COMMENT{at the beginning and the end of the sequence.}
        a           := |max|(i-|Wsize| + 1,0);
        b           := |min|(i,|length|(|Seq|)-1);
        |window|    := |subsequence|(|Seq|,a,b);
        |winprefix| := |best_prefix|(|window|);
        \IF |winprefix|.|score|>|Tres| 
        \THEN |start| := 0;
              |end|   := |winprefix|.|end|;
              |lim|   := |end|;
              \FOR j := 0 \TO |lim|-3 \DO
                |suffix| := |subsequence|(|window|,j,|lim|);
                |prefix| := |best_prefix|(|suffix|);
                \IF |prefix|.|score| > |score|
                \THEN |score| := |prefix|.|score|;
                      |start| := j;
                      |end|   := j + |prefix|.|end|;
                \FI
              \OD;
              |append|(|Res|,(a + |start|,a + |end|))
        \FI
    \OD
    \END
\ENDPROC
\END
\end{program}

\section{New Algorithm}

The original DUST algorithm is not symmetric with respect to taking reverse
complements. The new algorithm is symmetric and in all our tests, it masks 
no more than 1.7\% more bases than the original DUST.

Any interval selected by the original algorithm for masking has 
the following properies:

\begin{enumerate}
\item its length is less than or equal to the window length $W$;
\item its score is greater than the score threshold $T$;
\item scores of all its subintervals are at most as high as its own score.
\end{enumerate}

From now on, intervals of the input sequence satisfying properties 1--3 above
are called {\em perfect}.

The new symmetric DUST algorithm finds and masks {\em all} perfect intervals in the input 
sequence. First a simple algorithm to find all perfect intervals is described.
The next section presents some optimizations that result in significant 
performance improvement.

The new algorithm maintains the following data structures:

\begin{itemize}
\item a sliding window (subsequence) of length $W$ (or less if at the 
      beginning of the sequence);
\item a list $P$ of all perfect intervals within the sliding window and 
      their scores, sorted by their left ends.
\end{itemize}

When the sliding window shifts by one character the following steps are 
performed (note that any new perfect interval must be a suffix of the 
sliding window):

\begin{enumerate}
\item remove from $P$ all items that are not in the sliding window (all of them
      are in the beginning of $P$);
\item for each suffix of the sliding window with the score $S$ higher than $T$,
      if $S$ is larger than the maximum of the scores of elements in $P$ covered 
      by that suffix, then add the suffix to $P$, since it is necessarily a 
      new perfect interval.
\end{enumerate}

Below is the new algorithm in pseudocode.

\begin{program}
\BEGIN
\TYPE |perfect| := \TUPLE ( |start| : |int|,\ |end| : |int|,\ |score| : |real| );
\PROC |process_window|( 
    \tab |window|     : |sequence|,
         |Tres|       : |real|, 
         |wstart|     : |int|,
         \VARP |Perf| : \LIST \OF |perfect| ) \untab\BODY
    \BEGIN
    \VARP |counts|           : \ARRAY [0..63] \OF |int|;
    \VARP i, t, |len|, |sum| : |int|;
    \VARP |max_score|        : |real|;
    \VARP |curr_perfect|     : |list_iter|;
    \VARP |elem|             : |perfect|;
    \FOR i:=0 \TO 63 \DO |counts|[i] := 0 \OD;
    |curr_perfect| := |last_iter|(|Perf|);
    |max_score|    := 0;
    |len|          := 0;
    |sum|          := 0;
    \FOR i := |length|(|window|) - 3 \TO 0 \STEP -1 \DO
        t           := |triplet|(|window|, i);
        |sum|       := |sum| + |counts|[t];
        |counts|[t] := |counts|[t] + 1;
        \IF |len| > 0 \AND |sum|/|len| > |Tres|
        \THEN
            \WHILE |curr_perfect| \ne \NIL 
                   \AND |list_elem|(|curr_perfect|).|start| \ge i + |wstart| \DO
                |elem| := |list_elem|(|curr_perfect|);
                \IF |max_score| < |elem|.|score| 
                \THEN |max_score|       := |elem|.|score|;
                \FI
                |current_prefect| := |prev|(|curr_perfect|)
            \OD
            \COMMENT{In the following comparison it is important}
            \COMMENT{to consider the intervals with the score}
            \COMMENT{equal to |max\_score|. The definition of a}
            \COMMENT{perfect interval allows for proper}
            \COMMENT{subintervals of equal score.}
            \IF |sum|/|len| \ge |max_score|
            \THEN |max_score|    := |sum|/|len|;
                  |curr_perfect| := |insert|(
                    \qquad|curr_perfect|,
                    \qquad(
                    \qquad\ \quad i + |wstart|,
                    \qquad\ \quad|length|(|window|) - 1 + |wstart|,
                    \qquad\ \quad|max_score|
                    \qquad))
            \FI
        \FI
        |len| := |len| + 1
    \OD
    \END
\ENDPROC
\PROC |sdust|(
    \tab |Seq|       : |sequence|,
         |Tres|      : |real|,
         |Wsize|     : |int|,
         \VARP |Res| : |list_of_integer_pairs| )\untab\BODY
    \COMMENT{ Results are returned in |Res|. }
    \COMMENT{ |Perf| is initially empty. }
    \BEGIN
    \VARP |Perf|     : \LIST \OF |perfect|;
    \VARP i, |start| : |int|;
    \VARP |window|   : |sequence|;
    \FOR i:=3 \TO |length|(|Seq|) - 1 \DO
        |start|  := |max|(i-|Wsize|+1,0);
        |window| := |subsequence|(|Seq|, |start|, i);
        \WHILE \NOT |empty|(|Perf|) \AND |head|(|Perf|).|start| < |start| \DO
            |append|(|Res|,|head|(|Perf|).|start|,|head|(|Perf|).|end|);
            |pop_front|(|Perf|)
        \OD
        |process_window|( |window|, |Tres|, |start|, |Perf| )
    \OD
    \END
\ENDPROC
\END
\end{program}

\section{Optimization}

For the long-used current threshold of 20, window size of 64, and genomic test 
sets we have considered it is possible to partially or completely eliminate 
score computation for most windows.  To do this we estimate the length of the 
longest suffix of the window that is guaranteed to be free of perfect intervals. 
Then using the length of such suffix we can find a sufficient condition for the 
whole window to be free of perfect intervals.

Let |w| be a window of length $l+2$. For each $i \ge 0$, let $n_i$ be the number 
of triplet values that occur $i$ times in |w|. Let $|maxocc(w)|$ be the largest positive 
integer, such that there is a triplet value that occurs $|maxocc(w)|$ times in |w|.  Then 
the following equalities hold.
\begin{equation}\label{eql}
l = \sum_{i=0}^{|maxocc(w)|}{n_ii}
\end{equation}
\begin{equation}\label{eqS}
(l-1)S(|w|) = \sum_{i=0}^{|maxocc(w)|}{n_i\frac{i(i-1)}{2}}
\end{equation}

We are looking for the largest integer |threshocc| such that if $|maxocc(w)| \le |threshocc|$
then we can guarantee that
\begin{equation}\label{eqcond}
S(|w|) \le T{\rm.}
\end{equation}

Using (\ref{eql}) and (\ref{eqS}), (\ref{eqcond}) can be rewritten in the following
form:
\begin{equation}\label{eqcondone}
T \le \sum_{i=0}^k{n_i\left(\frac{(2T+1)i - i^2}{2}\right)}{\rm .}
\end{equation}

\begin{lemma}\label{lemmaone}
Let |w| be a window and |maxocc(w)| defined as above. If 

\begin{equation}\label{eqsuff}
|maxocc(w)| \le 2T
\end{equation}
for some $T \ge 1$ then $S(|w|) \le T$.
\end{lemma}

\proof Let $M = |maxocc(w)|$. We can assume that $M > 1$, because if every triplet appears at most once 
in |w|, then $S(|w|) = 0$. If $M \le 2T$ then 
$(2T + 1)i - i^2 = i( 2T + 1 - i ) \ge i \ge 0$ for every $i \le M$ . So every
element of the sum in the right side of (\ref{eqcondone}) is non negative.
Also $n_M \ge 1$ by definition of M. Therefore the right side of 
(\ref{eqcondone}) is greater or equal to 
${1\over2}n_M\left((2T + 1)M - M^2\right)$, and, to prove the lemma, it is 
sufficient to show that ${1\over2}\left((2T + 1)M - M^2\right) \ge T$ or,
equivalently, $(2T + 1)M - M^2 - 2T \ge 0$. The left side of the last inequality
can be written as $(M - 1)(2T - M)$ and, since $M > 1$ and $M \le 2T$, it is
indeed non negative. $\Box$

The bound in Lemma \ref{lemmaone} is tight. If |w| consists of $2T + 1$
identical triplets, then $|maxocc(w)| = |threshocc| + 1$ and $S(|w|) = T + 1 > T$.
This means that $|threshocc| = 2T$.

Previous discussion suggests one optimization of the symmetric DUST algorithm.
If, for each window |w|, we could maintain its longest suffix $|w|_1$ such that
(\ref{eqsuff}) holds for $|w|_1$, then we could skip partial score computations
for that suffix.

For a window |w| containing $l$ triplets, let $L(|w|)$ be the number of triplets 
in the longest suffix of |w| that satisfies~(\ref{eqsuff}). We want to find a 
condition (in terms of $L(|w|)$) such that processing of |w| can be omitted in the 
symmetric DUST procedure. In order to do this it is sufficient to make sure that 
inequality $S(|w|_j) \le T$ holds for every suffix $|w|_j$ of |w| containing $j$ 
triplets ($L(|w|) < j \le l$) or, equivalently
\begin{equation}\label{eqsuffone}
\sum_{t=0}^{63}{\frac{m_{t,j}(m_{t,j} - 1)}{2}} \le T(j - 1),%
\quad \forall j : L(|w|) < j \le l,
\end{equation}
where $m_{t,j}$ is the number of times the triplet value $t$ appears in 
$|w|_j$. We call the left parts of inequalities (\ref{eqsuffone}) $s_j$.

\begin{lemma}\label{lemmatwo}
If $s_l \le L(|w|)T$ then $S(|w|) \le T$.
\end{lemma}

\proof It is enough to show that if $s_l \le L(|w|)T$ then inequalities
(\ref{eqsuffone}) hold true. 

From the definition of $m_{t,j}$ it is clear that 
$m_{t,j} \le m_{t,l}$ for all triplet values $t$ and all $j \le l$. Therefore
$s_j \le s_l$ for $j \le l$. On the other hand $T(j-1) \ge TL(|w|)$ for 
$L(|w|) < j \le l$. So if $s_l \le L(|w|)T$, then for all $j$ such that 
$L(|w|) < j \le l$ the following is true
$$
\sum_{t=0}^{63}{\frac{m_{t,j}(m_{t,j} - 1)}{2}} = s_j \le s_l \le L(|w|)T \le T(j-1)
$$
which means that inequalities (\ref{eqsuffone}) hold. $\Box$

\subsection{Maintaining Window Suffix Information}

The following information is maintained by the symmetric DUST algorithm
implementation in order to implement the optimizations described above:

\begin{itemize}
\item the window suffix |suff|(|w|) of the current window |w| defined as
follows: let $k = \lfloor 2T \rfloor$ as before, then |suff|(|w|) is the
longest suffix such that every triplet appears no more than
$k$ times in |suff|(|w|).
\item $s_l$ and $s_L$ for each window |w|, where $l$ is the number of triplets 
in |w|, $L$ is the number of triplets in |suff|(|w|) defined above; these values 
are called {\em outer\_sum} and {\em inner\_sum} in the code;
\item the counts of each triplet value in |w| and |suff|(|w|); in the code these 
values are called, correspondingly, {\em outer\_counts} and {\em inner\_counts}.
\end{itemize}

Every time a window slides one letter to the right |push_triplet|() and
|pop_triplet|() functions are called. These functions keep the the data 
structures consistent when adding or removing a triplet from the window. In the 
following code $|thresocc| = \lfloor 2T \rfloor$, and |sf| is the position of the start 
of the window suffix.

\begin{program}
\BEGIN
\TYPE |counts_type| := \ARRAY[0..63] \OF |int|;
\PROC |push_triplet|( 
    \tab |wstart|, |threshocc|                : |int|,
         \VARP |sf|, |outer_sum|, |inner_sum| : |int|,
         \VARP |triplets|                     : |dequeue| \OF |int|, 
         \VARP |outer_counts|, |inner_counts| : |counts_type| ) \untab\BODY
    \BEGIN
    \VARP t : |int|;
    t                 := |last|(|triplets|);
    |outer_sum|       := |outer_sum| + |outer_counts|[t];
    |outer_counts|[t] := |outer_counts|[t] + 1;
    |add_thres_info|(|triplets|,t,|thresocc|,|wstart|,|sf|,|inner_sum|,|inner_counts|)
    \END
\ENDPROC
\PROC |pop_triplet|( 
    \tab\VARP |triplets|                     : |dequeue| \OF |int|, 
        \VARP |outer_sum|, |inner_sum|       : |int|,
        \VARP |outer_counts|, |inner_counts| : |counts_type| ) \untab\BODY
    \BEGIN
    \VARP t : |int|;
    t := |triplets|[0];
    \IF |inner_counts|[t] = |outer_counts|[t] 
    \THEN |rem_thres_info|(t, |inner_sum|, |inner_counts| )
    \FI
    |outer_counts|[t] := |outer_counts|[t] - 1;
    |outer_sum|       := |outer_sum| - |outer_counts|[t]
    \END
\ENDPROC
\END
\end{program}

Procedures |add_thres_info|() and |rem_thres_info|() are responsible for maintaining
the |inner_sum| and |inner_counts|. 

\begin{program}
\BEGIN
\TYPE |counts_type| := \ARRAY[0..63] \OF |int|;
\PROC |add_thres_info|(
    \tab|triplets|              : |dequeue| \OF |int|, 
        t, |thresocc|, |wstart|          : |int|,
        \VARP |sf|, |inner_sum| : |int|, 
        \VARP |inner_counts|    : |counts_type|) \untab\BODY
    \BEGIN
    \VARP |offset| : |int|;
    |inner_sum|       := |inner_sum| + |inner_counts|[t];
    |inner_counts|[t] := |inner_counts|[t] + 1;
    |offset|          := |sf| - |wstart|;
    \IF |inner_counts|[t] > |thresocc|
    \THEN
        \COMMENT{Updating the current suffix. The following}
        \COMMENT{loop removes all triplets from the front of the}
        \COMMENT{suffix up to and including the first triplet with}
        \COMMENT{value |t|. This makes the internal count of |t| less}
        \COMMENT{than |threshocc|.}
        \DODO
            |rem_thres_info|(t, |inner_sum|, |inner_counts|);
            |sf|     := |sf| + 1;
            |offset| := |offset| + 1
        \DOWHILE |triplets|[|offset| - 1] \not= t
    \FI
    \END
\ENDPROC
\PROC |rem_thres_info|( 
    \tab t                    : |int|, 
         \VARP |inner_sum|    : |int|, 
         \VARP |inner_counts| : |counts_type| ) \untab\BODY
    \BEGIN
    |inner_counts|[t] := |inner_counts|[t] - 1;
    |inner_sum|       := |inner_sum| - |inner_counts|[t]
    \END
\ENDPROC
\END
\end{program}

\subsection{Optimized Algorithm}

Below is the pseudocode of the optimized algorithm. 

\begin{program}
\BEGIN
\TYPE |perfect|     := \TUPLE ( |start| : |int|,\ |end| : |int|,\ |score| : |real| );
\TYPE |counts_type| := \ARRAY[0..63] \OF |int|;
\PROC |process_window|( 
\tab \VARP |triplets|                     : |dequeue| \OF |int|,
     |Tres|                               : |real|,
     |Wsize|, |wstart|                    : |int|,
     \VARP |sf|, |outer_sum|, |inner_sum| : |int|,
     \VARP |outer_counts|, |inner_counts| : |counts_type|,
     \VARP |Perf|                         : \LIST \OF |perfect| )\untab\BODY
    \BEGIN
    \VARP |counts|                         : |counts_type|;
    \VARP i, |threshocc|, |len|, |sum|, |suffix_len| : |int|;
    \VARP |curr_perfect|                   : |list_iter|;
    \VARP |elem|                           : |perfect|;
    |threshocc| := |floor|(2*|Tres|);
    \IF |length|(|triplets|) > |Wsize| - 1
    \THEN |pop_triplet|( 
        \qquad |triplets|, 
        \qquad |outer_sum|, |inner_sum|, 
        \qquad |outer_counts|, |inner_counts| )
    \FI
    |push_triplet|(
        \qquad |wstart|, |threshocc|, |sf|, 
        \qquad |outer_sum|, |inner_sum|, 
        \qquad |triplets|, 
        \qquad |outer_counts|, |inner_counts| );
    |suffix_len| := |length|(|triplets|) - (|sf|-|wstart|);
    \IF |outer_sum| > |suffix_len|*|Tres|
    \THEN
        \FOR i:=0 \TO 63 \DO |counts|[i] := |inner_counts|[i] \OD;
        |curr_perfect| := |last_iter|(|Perf|);
        |max_score|    := 0
        |len|          := |suffix_len|;
        |sum|          := |inner_sum|;
        \FOR i := |length|(|triplets|) - 1 - |suffix_len| \TO 0 \STEP -1 \DO
            t           := |triplets|[i];
            |sum|       := |sum| + |counts|[t];
            |counts|[t] := |counts|[t] + 1;
            \IF |sum|/|len| > |Tres|
            \THEN
                \WHILE |curr_perfect| \ne \NIL 
                       \AND |list_elem|(|curr_perfect|).|start| \ge i + |wstart| \DO
                    |elem| := |list_elem|(|curr_perfect|);
                    \IF |max_score| < |elem|.|score| 
                    \THEN |max_score|       := |elem|.|score|;
                    \FI
                    |current_prefect| := |prev|(|curr_perfect|)
                \OD
                \COMMENT{In the following comparison it is important}
                \COMMENT{to consider the intervals with the score}
                \COMMENT{equal to |max\_score|. The definition of a}
                \COMMENT{perfect interval allows for proper}
                \COMMENT{subintervals of equal score.}
                \IF |sum|/|len| \ge |max_score|
                \THEN |max_score| := |sum|/|len|;
                      |curr_perfect| := |insert|(
                          \qquad|curr_perfect|,
                          \qquad(
                          \qquad\ \quad i + |wstart|,
                          \qquad\ \quad|length|(|triplets|) + 1 + |wstart|,
                          \qquad\ \quad|max_score|
                          \qquad))
                \FI
            \FI
            |len| := |len| + 1
        \OD
    \FI
    \END
\ENDPROC
\PROC |sdust|(
    \tab |Seq|       : |sequence|,
         |Tres|      : |real|,
         |Wsize|     : |int|,
         \VARP |Res| : |list_of_integer_pairs| )\untab\BODY
    \COMMENT{ Results are returned in |Res|. }
    \COMMENT{ |Res| is initially empty. }
    \BEGIN
    \VARP i, |sf|, |start|, |outer_sum|, |inner_sum| : |int|;
    \VARP |outer_counts|, |inner_counts|             : |counts_type|;
    \VARP |Perf|                                     : \LIST \OF |perfect|;
    \VARP |triplets|                                 : |dequeue| \OF |int|;
    |sf|        := 0;
    |inner_sum| := 0;
    |outer_sum| := 0;
    \FOR i := 0 \TO 63 \DO
        |inner_counts|[i] := 0;
        |outer_counts|[i] := 0
    \OD
    \FOR i:=2 \TO |length|(|Seq|) - 1 \DO
        |start|        := |max|(i-|Wsize|+1,0);
        |push_back|( |triplets|, |triplet|(|Seq|, i-2) );
        \WHILE \NOT |empty|(|Perf|) \AND |head|(|Perf|).|start| < |start| \DO
            |append|(|Res|,(|head|(|Perf|).|start|,|head|(|Perf|).|end|));
            |pop_front|(|Perf|)
        \OD
        |process_window|( |triplets|, |Tres|, |Wsize|, |start|, |sf|, 
        \qquad |outer_sum|, |inner_sum|, |outer_counts|, |inner_counts|, |Perf| )
    \OD
    \END
\ENDPROC
\END
\end{program}

\section{Performance}

All tests were performed on a dual Pentium 4 Xeon 3.2 Ghz
Linux workstation with 4 Gb of RAM running Linux kernel 2.4.23.
Applications were compiled with GNU C/C++ compilers 
version 3.4.0 with full optimization (-O3 -fomit-frame-pointer -ffast-math).

The following table shows running times in seconds of the original DUST
and optimized symmetric DUST when masking genomes of {\it Drosophila melanogaster}
and {\it Homo sapiens}.
For each test 3 runs were performed and the average time is reported
in the table.

\vskip 1cm
\begin{tabular}{|l|r|r|r|}
\hline
 & original & symmetric & ratio of symmetric DUST time\\
 & DUST\hfill\hfill     & DUST\hfill\hfill      & to original DUST time \hfill\hfill \\
\hline
{\it D. melanogaster} & 113.27   & 32.47 & .2870 \\
\hline
{\it H. sapiens} chr1  & 231.86 & 65.73 & .2835 \\
{\it H. sapiens} chr2  & 245.16 & 68.36 & .2788 \\
{\it H. sapiens} chr3  & 195.81 & 54.58 & .2787 \\
{\it H. sapiens} chr4  & 210.38 & 53.65 & .2550 \\
{\it H. sapiens} chr5  & 197.54 & 50.30 & .2546 \\
{\it H. sapiens} chr6  & 184.34 & 47.90 & .2598 \\
{\it H. sapiens} chr7  & 162.33 & 45.05 & .2775 \\
{\it H. sapiens} chr8  & 156.36 & 40.98 & .2621 \\
{\it H. sapiens} chr9  & 130.81 & 33.84 & .2587 \\
{\it H. sapiens} chr10 & 144.12 & 38.02 & .2638 \\
{\it H. sapiens} chr11 & 133.29 & 36.61 & .2747 \\
{\it H. sapiens} chr12 & 135.87 & 37.60 & .2767 \\
{\it H. sapiens} chr13 & 101.65 & 27.24 & .2680 \\
{\it H. sapiens} chr14 & 92.81  & 24.90 & .2683 \\
{\it H. sapiens} chr15 & 85.92  & 24.01 & .2794 \\
{\it H. sapiens} chr16 & 104.46 & 24.99 & .2392 \\
{\it H. sapiens} chr17 & 102.5  & 25.55 & .2493 \\
{\it H. sapiens} chr18 & 76.56  & 22.70 & .2965 \\
{\it H. sapiens} chr19 & 68.59  & 19.02 & .2773 \\
{\it H. sapiens} chr20 & 62.44  & 17.13 & .2743 \\
{\it H. sapiens} chr21 & 37.12  & 10.09 & .2718 \\
{\it H. sapiens} chr22 & 38.21  & 10.18 & .2664 \\
{\it H. sapiens} chrX  & 167.67 & 62.98 & .3756 \\
{\it H. sapiens} chrY  & 32.91  & 8.33  & .2531 \\
%d.{\it melanogaster} & 113.27   & 90.17 & 32.47 & 71.3\% \\
%run 1 & 1m52.839s & 1m29.546s & 0m32.335s & \\
%run 2 & 1m53.221s & 1m30.962s & 0m32.686s & \\
%run 3 & 1m53.750s & 1m30.006s & 0m32.380s & \\
\hline
\end{tabular}

\vskip 1cm
The next table shows the number of bases masked by each algorithm
in genomes of d.{\it melanogaster} and h.{\it sapiens}.

\begin{tabular}{|l|r|r|r|}
\hline
 & original DUST & symmetric DUST & percentage of increase \\
\hline
{\it D. melanogaster} & 5278484 & 5368015 & 1.70 \% \\
\hline
{\it H. sapiens} chr1  & 9827477  & 9962990  & 1.39 \% \\
{\it H. sapiens} chr2  & 10396020 & 10534797 & 1.33 \% \\
{\it H. sapiens} chr3  & 8174058  & 8291266  & 1.43 \% \\
{\it H. sapiens} chr4  & 8501155  & 8612129  & 1.31 \% \\
{\it H. sapiens} chr5  & 7606117  & 7711431  & 1.38 \% \\
{\it H. sapiens} chr6  & 7327405  & 7429406  & 1.39 \% \\
{\it H. sapiens} chr7  & 7134998  & 7231658  & 1.35 \% \\
{\it H. sapiens} chr8  & 6209036  & 6292534  & 1.34 \% \\
{\it H. sapiens} chr9  & 5181032  & 5251123  & 1.35 \% \\
{\it H. sapiens} chr10 & 6008077  & 6086515  & 1.31 \% \\
{\it H. sapiens} chr11 & 5467439  & 5539418  & 1.32 \% \\
{\it H. sapiens} chr12 & 5815186  & 5895050  & 1.37 \% \\
{\it H. sapiens} chr13 & 4319532  & 4377601  & 1.34 \% \\
{\it H. sapiens} chr14 & 3775868  & 3828803  & 1.40 \% \\
{\it H. sapiens} chr15 & 3539237  & 3587821  & 1.37 \% \\
{\it H. sapiens} chr16 & 3899069  & 3946925  & 1.23 \% \\
{\it H. sapiens} chr17 & 3766104  & 3817441  & 1.36 \% \\
{\it H. sapiens} chr18 & 3259307  & 3303535  & 1.36 \% \\
{\it H. sapiens} chr19 & 3326872  & 3368114  & 1.24 \% \\
{\it H. sapiens} chr20 & 2619954  & 2654471  & 1.32 \% \\
{\it H. sapiens} chr21 & 1631269  & 1651742  & 1.26 \% \\
{\it H. sapiens} chr22 & 1630297  & 1651307  & 1.29 \% \\
{\it H. sapiens} chrX  & 6954440  & 7044967  & 1.30 \% \\
{\it H. sapiens} chrY  & 1534226  & 1547697  & 0.88 \% \\
\hline
\end{tabular}

\section{Type and Function definitions}\label{secpseudo}

\subsection{Types}

|int| is the type used to represent integers

|real| is the type used to represent real numbers

|vector| is a sequence of elements of the same type. It is different from
\ARRAY in that |vector| can have variable length. |vector| supports random
access of its elements by index. Indices start at $0$.

|dequeue| is a variable length sequence of elements of the same type supporting
efficient random access of its elements as well as efficient append/remove to/from
both ends of the sequence.

|tuple| is a sequence of elements of possibly different types that has a fixed
length. Elements of a tuple are named. If |tup| is a tuple with an element named
|start|, then that element can be accessed as |tup|.|start|. In the pseudocode
an instance of a tuple having, e.g. 3 elements |a|, |b|, and |c| is written
as (|a|, |b|, |c|).

|list| is a doubly linked list of elements of the same type. The number of
elements in |list| is not fixed. |list| does not support random access of its
elements, but supports efficient forward and backwards traversal.

|list_iter| iterator type used to traverse a list.

\TYPE |sequence| : |vector| \OF $\{ \mathbf{A}, \mathbf{C}, \mathbf{G}, \mathbf{T} \}$

\TYPE |integer_pair| : |tuple|( |first| : |int|, |second| : |int| )

\TYPE |list_of_integer_pairs| : |list| \OF |integer_pair|

\subsection{Functions and Procedures}

\PROC |append|( |L| : |list| \OF |elem_type|, |elem| : |elem_type| ) \ENDPROC \\
Appends a new element to the back of the list.

\FUNCT |floor|( |r| : |real| ) : |int| \ENDFUNCT \\
Returns $\lfloor r \rfloor$.

\FUNCT |head|( |L| : |list| \OF |elem_type| ) : |elem_type| \ENDFUNCT \\
Returns the first element of the list.

\PROC |insert|( |iter| : |list_iter|, |elem| : |elem_type| ) \ENDPROC \\
Inserts a new element into the list in front of the one pointed to by the iterator.

\FUNCT |last_iter|( |L| : |list| ) : |list_iter| \ENDFUNCT \\
Returns an iterator pointing to the last element of the list.

\FUNCT |length|( |C| : |container_type| ) : |int| \ENDFUNCT\\
Generic function that returns the number of elements in a container
(e.g. vector, sequence, etc.).

\FUNCT |list_elem|( |iter| : |list_iter| ) : |elem_type| \ENDFUNCT \\
Returns the element of the list pointed to by the iterator.

\FUNCT |max|( |a| : |int|, |b| : |int| ) : |int| \ENDFUNCT \\
Returns tha maximum of two integer values.

\PROC |pop_front|( |C| : |container_type| ) \ENDPROC \\
Removes the first element from sequential container.

\FUNCT |prev|( |iter| : |list_iter| ) : |list_iter| \ENDFUNCT \\
Gets the list iterator preceding the given iterator.

\PROC |push_back|( |C| : |container_type|, |elem| : |elem_type| ) \ENDPROC \\
Appends an element to a sequential container.

\FUNCT |subsequence|( |S| : |sequence|, |start| : |int|, |end| : |int| ) : |sequence| \ENDFUNCT\\
Returns a subsequence of |S|. |start| is the index of the first element of
the subsequence. |end| is the index of the last element of the subsequence.

\FUNCT |triplet|( |S| : |sequence|, |i| : |int| ) : |int|\ENDFUNCT\\
Returns a triplet value of the triplet that starts at index |i| in |S|.

\end{document}

