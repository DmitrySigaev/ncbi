\documentclass[english]{article}
\usepackage{verbatim}
\usepackage[fancy]{latex2man}

\setVersion{3.101}
\begin{document}

\begin{Name}{1}{oligoFAR}{NCBI/NLM/NIH}{version \Version}{oligoFAR -- global alignment of single or paired short read}
	\Prog{oligoFAR} is a flexible short read alignment tool targeted on exploring of data properties.
	It allows flexible read structure (reads with predefined gap locations), 
	supports 1 and 4-channel quality scores, colorspace and mapping of data produced after bisulfite-treatment of dna.
\end{Name}

\section{Synopsis}
	\Prog{oligofar} \oOptoArg{--help}{=brief|full|extra} \oOpt{-hV} \oOptArg{-C}{~cfgfile}
	\oOpt{--pass0} \oArg{pass-options} \oOptoArg{--pass1}{~pass-options}
	\oOptArg{-i}{~infile} \OptArg{-d}{~dbfile} \oOptArg{-b}{~snpdb} \oOptArg{-g}{~guide} \oOptArg{-v}{~feat} \oOptArg{-l}{~gilist}
	\oOptArg{-1}{~reads1} \oOptArg{-2}{~reads2} \oOptArg{-o}{~outfile} \oOptArg{-y}{~seq-id} \oOptArg{-c}{~yes|no} \oOptArg{-q}{~0|1|4}
	\oOptArg{-1}{-0~value|+char} \oOptArg{-O}{~flags} \oOptArg{-B}{~bsize} \oOptArg{--batch-range}{=min\Lbr-max\Rbr}
	\oOptArg{--NaHSO3}{=yes|no} \oOptArg{-s}{~1|2|3} \oOptArg{--hash-bitmap-file}{=hbmfile}
	\oOptArg{-I}{~score} \oOptArg{-M}{~score} \oOptArg{-G}{~score} \oOptArg{-Q}{~score} \oOptArg{-x}{~score}
	\oOptArg{-p}{~pctid} \oOptArg{-u}{~topcnt} \oOptArg{-t}{~toppct} \oOptArg{-R}{~value}
	\oOptArg{-U}{~version} \oOptArg{-L}{~memsz} \oOptArg{-T}{~yes|no} \oOptArg{--fasta-parse-ids}{=yes|no}
	
	where \emph{pass-options} are:

	\oOptArg{-w}{~window\Lbr/word\Rbr} \oOptArg{-k}{~pos\Lbr,...\Rbr} \oOptArg{-f}{~bases} \oOptArg{-r}{~bases} 
	\oOptArg{-S}{~stride} \oOptArg{-H}{~bits} \oOptArg{-N}{~count} \oOptArg{-a}{~amb} \oOptArg{-A}{~amb} \oOptArg{-P}{~score} 
	\oOptArg{-F}{~simpl} \oOptArg{-n}{~mismatch} \oOptArg{-e}{~len} \oOptArg{-j}{~len} \oOptArg{-J}{~len} \oOptArg{-E}{~cnt}
	\oOptArg{-K}{~pos} \oOptArg{-X}{~value} \oOptArg{-Y}{~value} \oOptArg{--longest-ins}{=value} 
	\oOptArg{--longest-del}{=value} \oOptArg{--max-inserted}{=value} \oOptArg{--max-deleted}{=value} 
	\oOptArg{--add-splice}{=pos(\Lbr min:\Rbr max)} \oOptArg{-D}{~min\Lbr-max\Rbr} \oOptArg{-m}{~len}

\section{Examples}

	\Prog{oligofar} \OptArg{-i}{~pairs.tbl} \OptArg{-d}{~contigs.fa} \OptArg{-l}{~gilist} \OptArg{-g}{~guide.sam}
	\OptArg{-w}{~20/12} \OptArg{-B}{250000} \OptArg{-H}{22} \OptArg{-n}{2} \OptArg{-p}{90} \OptArg{-D}{100-500}
	\OptArg{-m}{50} \OptArg{-R}{p} \OptArg{-L}{16G} \OptArg{-o}{~output.sam}

\section{Input Format Options}

	Following combinations of input format and data flags are allowed:

	\begin{enumerate}
	\item with column file:\\
		\OptArg{-q}{0} \OptArg{-i}{~input.col} \OptArg{-c}{~no}\\
		\OptArg{-q}{1} \OptArg{-i}{~input.col} \OptArg{-c}{~no}\\
		\OptArg{-q}{0} \OptArg{-i}{~input.col} \OptArg{-c}{~yes}
	\item with fasta or fastq files:\\
		\OptArg{-q}{0} \OptArg{-1}{~reads1.fa} \oOptArg{-2}{~reads2.fa} \OptArg{-c}{~no}\\
		\OptArg{-q}{0} \OptArg{-1}{~reads1.fa} \oOptArg{-2}{~reads2.fa} \OptArg{-c}{~yes}\\
		\OptArg{-q}{1} \OptArg{-1}{~reads1.faq} \oOptArg{-2}{~reads2.faq} \OptArg{-c}{~no}
	\item with Solexa 4-channel data:\\
		\OptArg{-q}{4} \OptArg{-i}{~input.id} \OptArg{-1}{~reads1.prb} \oOptArg{-2}{~reads2.prb} \OptArg{-c}{~no}
	\end{enumerate}

	See options and file formats for more info

\section{Changes}
	Following parameters are now, have changed or have disappeared:\\
	\begin{itemize}
    \item in version 3.25: -n, -w, -N, -S, -x, -f, -R 
    \item in version 3.26: -n, -w, -N, -z, -Z, -D, -m, -S, -x, -f, -k
    \item in version 3.27: -n, -w, -e, -H, -S, -a, -A, --pass0, --pass1
    \item in version 3.28: -y, -R, -N
    \item in version 3.29: --NaHSO3 (Development)
    \item in version 3.91: -X -Y -r -O --NaHSO3 
    \item in version 3.98: -x -g -O -B
    \item in version 3.100: -v 
    \item in verison 3.101: -i -1 -2 -q -O
	\end{itemize}

\section{Description}
    Performs global alignments of multiple single or paired short reads 
    with noticeable error rate to a genome or to a set of transcripts 
    provided in a blast-db or a fasta file.

    Reads may be provided as UIPACna base calls, possibly accompanied 
    with phrap scores (referred below as 1-channel quality scores), 
    or as 4-channel Solexa scores. Input file format is described 
    below in section FILE FORMATS.

    Output of srsearch (referred below as guide-file) or of a similar 
    program which performs exact or nealy exact short read alignment
    may be used as input for oligoFAR to ignore processing of perfectly 
    matched reads, but format the matches to output in uniform with
    oligoFAR matches way.

    Input is processed by batches of size controlled by option \Opt{-B}. 
    Reads to match are hashed (one window (unless option -N is used) per read, 
    preferrably at the 5' end) with a window size controlled by option \Opt{-w}. 
    Option \Opt{-n} controls how many mismatches are allowed within hashed values.  
    Option \Opt{-a} controls how many ambiguous bases withing a window of a read 
    may be hashed independently to mismatches allowed. Low quality 3' ends 
    of the reads may be clipped.  Low complexity (controlled by \Opt{-F} argument) 
    and low quality reads may be ignored.

    OligoFAR may use different implementations of the hash table (see \Opt{-H}):
    vector (uses a lot of memory, but is faster for big batches) and
    arraymap (lower memory requirements for smaller batches). 
    For vector \Opt{-L} should always be used and set to large value (GygaBytes).

    Database is scanned. If database is provided as blastdb, it is 
    possible to limit scan to a number of gis with option \Opt{-l}. If snpdb is
    provided, all variants of alleles are used to compute hash values, as well
    as regular IUPACna ambiguities of the sequences in database. Option \Opt{-A}
    controls maximum number of ambiguities in the same window.

    Alignments are seeded by hash and may be extended by Smith-Watermann
    algorithm (unless \OptArg{-X}{0} or \OptArg{-Y}{0} is used).

    Alignments are filtered (see \Opt{-p} option). For paired reads geometrical
    constraints are applied (reads of the same pair should be mutually 
    oriented according to \Opt{-R} option, distance is set by \Opt{-D} and \Opt{-m}
    options). Then hits ranked by score (hits of the same score have same
    rank, best hits have rank 0). Week hits or too repetitive hits are 
    thrown away (see \Opt{-t} and \Opt{-u} options).

    At the end of each batch both alignments produced by \Prog{oligoFAR} and
    alignments imported from guide-file which have passed filtering and
    ranking get printed to output file (if set) or stdout (see 
	FILE FORMATS for output format).

\subsection{Note}

    Since it is global alignment tool, independent runs against, say,
    individual chromosomes and run against full genome will produce different 
    results.

    To save disk space and computational resources, \Prog{oligoFAR} ranks hits by
    score and reports only the best hits and ties to the best hits. 
    In the two-pass mode tie hits may be incompletely reported -- in this 
    case only hits of same score as the best are guarranteed to appear in 
    output no matter what value of \Opt{-t} is set.

    Scores of hits reported are in percent to the best score theoretically
    possible for the reads. Scores of paired hits are sums of individual
    scores, so they may be as high as 200.
    
\subsection{Paired Reads}

	Pairs are looked-up constrained by following requirements: 
	\begin{itemize}
	\item relative orientation (geometry) which may be set by \Opt{--geometry} 
	   or \Opt{-R} (see section OPTIONS subsection ``Filtering and ranking options'')
    \item distance between lowest position of the two reads and highest
       position of the two reads one should be in range \Lbr~\$a~-~\$m;~\$b~+~\$m~\Rbr 
       where \$a, \$b and \$m are arguments of parameters \OptArg{-D}{~\$a-\$b} 
	   \OptArg{-m}{~\$m}.
	\end{itemize}

    If pair has no hits which comply constraints mentioned above, individual 
    hits for the pair components still will be reported. Also for each
    component unpaired hits better then the best paired hit will be reported.

    Paired reads have one ID per pair. Individual reads in this case do not
    have individual ID, although report provides info which component(s) of 
    the pair produce the hit.

\subsection{Sodium Bisulfite Treatment}
    
	To discover methylation state of DNA sodium bisulfite curation may be
    used before producing reads.  In order to simulate this procedure
    oligoFAR has special mode, which may be turned on by:
    
        \OptArg{--NaHSO3}{=true}

    It is advised to use longer words and windows in this mode for better
    performance. 
    
    This mode is not compatible with colorspace computations.

\subsection{Multipass Mode}
   
   By default oligoFAR aligns all reads just once, but if option \Opt{--pass1} is 
    used, oligoFAR switches to the two-pass mode. Parameters \Opt{-w}, \Opt{-n}, \Opt{-e}, \Opt{-H}, 
    and some other, preceeding \Opt{--pass1} or following \Opt{--pass0} affect first run, same 
    parameters when follow \Opt{--pass1} are for the second run.  For the second run 
    only reads (or pairs) having more mismatches or indels then allowed in 
    parameters for the first pass will be hashed and aligned. So using something 
    like:

	\Prog{oligofar} \Opt{--pass0} \OptArg{-w}{22/22} \OptArg{-n}{0} \OptArg{-e}{0} \Opt{--pass1} \OptArg{-w}{22/13} \OptArg{-n}{2} \OptArg{-e}{1} 

    will pick up exact matches first, and then run search with less strict 
    parameters only for those reads which did not have exact hits.

\subsection{Window, Stride and Word}
    When hashing, oligoFAR first chooses some region on the read (called window).
    If the stride is greater then 1, it extracts the "stride" windows with offset 
    of 1 to each other. Then each window gets variated (with mismatches and/or 
    indels). If word size is equal to window size, window is hashed as is. If 
    word size is smaller then window, two possibly overlapping words are created
    and added to hash: at the beginning of the window and at the end of the 
    window.

    Example for case \OptArg{-N}{~2} \OptArg{-w}{~13/9} \OptArg{-S}{~3} \OptArg{-e}{~0}:
    
	\begin{verbatim}
    +-------------+               hashed region 1 (should not exceed 32 bases)
                 +-------------+  hashed region 2 (should not exceed 32 bases)
    ACGTGTTGATGACTACTGATGATCTGATccat
    +-----------+                 window group 1
    ACGTGTTGATGAC                 window 1 \
     CGTGTTGATGACT                window 2  } stride = 3
      GTGTTGATGACTA               window 3 /
                 +-----------+    window group 2
                 TACTGATGATCTG    window 4 \
                  ACTGATGATCTGA   window 5  } stride = 3
                   CTGATGATCTGAT  window 6 /
    +-------+                     word size = 9
    ACGTGTTGA                     word 1 of window 1
        GTTGATGAC                 word 2 of window 1
     CGTGTTGAT                    word 1 of window 2
         TTGATGACT                word 2 of window 2
      GTGTTGATG                   word 1 of window 3
          TGATGACTA               word 2 of window 3
                 +-------+        word size = 9
                 TACTGATGA        word 1 of window 4
                     GATGATCTG    word 2 of window 4
                  ACTGATGAT       word 1 of window 5
                      ATGATCTGA   word 2 of window 5
                   CTGATGATC      word 1 of window 6
                       TGATCTGAT  word 2 of window 6
   \end{verbatim}

    If you allow indels to be hashed, hashed region is extended by 1 base. So 
    for window of 24 and stride 4 with indels allowed reads should be at least
    28 bases long (27 with no indels).

    Words when hashing are splet into the two parts: index and supplement. 
    Supplement can't have more then 16 bits. Index can't be more then 31 bits.
    Therefore maximal word size should fit in 47 bits which is 23 bases.


\section{Options}\label{secOptions}

\subsection{Service options}
\begin{description}
\item[\OptoArg{--help}{=brief|full|extended}~~\Opt{-h}] 
				Print help to stdout, finish parsing commandline, and then 
                exit with error code 0. Long version may accept otional 
                argument which specifies should be printed brief help version
                (synopsis), full version (without extended options) or extended
                options help.

                The output (except brief) contains current values taken for options, 
                so commandline arguments values which preceed \Opt{-h} or \Opt{--help} will
                be reflected in the output, for others default values will be
                printed.

\item[\Opt{--version}~~\Opt{-V}]
				Print current version and hash implementation, finish parsing
                commandline and exit with error code 0.

\item[\OptArg{--assert-version}{=version}~~\OptArg{-U}{~version}]
				If oligofar version is not that is specified in argument,
                forces oligofar to exit with error.  Every time this option 
                appears in commandline or in config file, the comparison is
                performed, so each config file may contain this check
                independently.  

\item[\OptArg{--test-suite}{=+|-}~~\OptArg{-T}{+|-}]
				Run internal tests for basic operations before doing anything
                else.
\item[\OptArg{--memory-limit}{=value}~~\OptArg{-L}{~value}]
				Set upper memory limit to given value (in bytes). Suffixes k,
                M, G are allowed. Default is 'all free RAM + cache + buffers'.
                Recommended value for VectorTable hash implementation is 14G or
                above.
\item[\OptoArg{--write-config}{=file}]
				Write current option set as config file to the given file or stdout; 
				terminate with exit code 0 after command line is parsed
\item[\OptArg{--config-file}{=file}~~\OptArg{-C}{~file}]
				Parse config file (may be used multiple times)
\end{description}

\subsection{File options}

\begin{description}
\item[\OptArg{--input-file}{=filename}~~\OptArg{-i}{~filename}]
    			For \OptArg{-q}{0} and \OptArg{-q}{1} read file with 2,3, 
				or 5 space-separated 
                columns (see FILE FORMATS); for \OptArg{-q}{4} read only first column 
                with read IDs.
\item[\OptArg{--fasta-file}{=filename}~~\OptArg{-d}{~filename}
				Sets database file name. If there exists file with given name
                extended with one of suffixes .nin, .nal -- the suffixed version 
                is opened as blastdb, otherwise the file itself is expected to
                contain sequences in FASTA format. 
\item[\OptArg{--guide-file}{=filename}~~\OptArg{-g}{~filename}]
				Sets guide file name. The file should have exactly the same order 
                of reads as input file does. There may be multiple hits per each
                read, or some reads may be skipped.
\item[\OptArg{--snpdb-file}{=filename}~~\OptArg{-b}{~filename}] 
				Sets SNP database filename. Snpdb is a file prepared by an
                oligofar.snpdb program.
\item[\OptArg{--hash-bitmap-file}{=filename}]
				Specifies file to use to check if given word appears in genome -- useful
				to minimize memory usage with errors allowed in hash words
\item[\OptArg{--feat-file}{=filename}~~\OptArg{-v}{~filename}] 
				Sets filename for feature file which is three column file
                containing subject sequence id, begin and end positions for
                regions within which scanning should be performed
\item[\OptArg{--output-file}{=filename}~~\OptArg{-o}{~filename}]
				Sets output file name.
\item[\OptArg{--gi-list}{=filename}~~\OptArg{-l}{~filename}]
				Sets file with list of gis to which scan of database should be
                limited. Works only if database is in blastdb.
\item[\OptArg{--only-seqid}{=seqid}~~\OptArg{-y}{~seqid}]
				Limits database lookup to this seqid. May appear multiple
                times -- then list of seqids is used. Does not work with \Opt{-l}.
                Comparison is pretty strict, so lcl| or .2 are required in
                'lcl|chr12' and 'NM_012345.2'. 
\item[\OptArg{--fasta-parse-ids}{=yes|no}]
				Specifies should oligofar try to parse IDs in input fasta or should 
				use them 'as-is' as plain strings. Default is 'no' to avoid parsing of 
				Seq-ids with ranges; may be changed to 'yes' to make only one ID of multiple
				appear on output
\item[\OptArg{--read-1-file}{=filename}~~\OptArg{-1}{~filename}]
				For \OptArg{-q}{0} read this file as fasta file of the reads sequences;
                for \OptArg{-q}{1} read this file as fastq file of the reads sequences and
                quality; for \OptArg{-q}{4} read file with 4-channel (Solexa) quality scores, 
                then this requires also -i for read IDs (which should go in
                the same order). All 1 sequence per read -- for paired reads,
                pair mates will be in the file specified by option \Opt{-2}.

\item[\OptArg{--read-2-file}{=filename}~~\OptArg{-2}{~filename}]
    			In case of paired reads contains pair mate data in the same
                order and the same format as in file specified by option \Opt{-1}.
\item[\OptArg{--quality-channels}{=cnt}~~\OptArg{-q}{0|1|4}]
				What data are expected on input: with \OptArg{-q}{0} it should be either 
                2 or 3 column file in -i or fasta file(s) in \Opt{-1}, \Opt{-2}; with \OptArg{-q}{1} 
                it should be either 5-column file in \Opt{-i} or fastq file(s) in
                \Opt{-1}, \Opt{-2}; with \OptArg{-q}{4} read IDs are read from \Opt{-i} and Solexa
                4-channel scores from \Opt{-1} and \Opt{-2}.
\item[\OptArg{--quality-base}{=value}~~\OptArg{-0}{~value}]
                Sets base value for character-encoded phrap quality scores,
                i.e. integer ASCII value or `+' followed by character which 
                corresponds to the phrap score of 0.
\item[\OptArg{--colorspace}{=+|-}~~\OptArg{-c}{+|-}]
				Input is in di-base colorspace encoding. Hashing and alignment 
                will be performed in the colorspace encoding.  Requires \OptArg{-q}{0}.
\item[\OptArg{--NaHSO3}{=+|-}]
                Turns sodium bisulfite treatment simulation mode on or off.
                Make sure to use reasonable value for \Opt{-A}.
\item[\OptArg{--output-flags}{=-huxmtderzZp}~~\OptArg{-O}{-huxmtderzZp}]
                Controls what types of records should be produced on output.
                See OUTPUT FORMAT section below. Flags stand for:
	
	\begin{Description}[w]
	\item[e] empty line
    \item[u] unmapped reads
    \item[m] "more" lines (there were more hits of weaker rank, dropped)
    \item[x] "many" lines (there were more hits of this rank and below)
    \item[t] terminator (there were no more hits except reported)
    \item[a] alignment -- output alignment details in comments 
    \item[d] differences -- output positions of misalignments
    \item[r] print raw scores, rather then relative to the best
    \item[h] print all hits above threshold before ranking
	\item[z] output in SAM 1.2 format (clears flag Z)
	\item[Z] output in native format (clears flag z)
	\item[p] print unpaired hits only if no paired found
	\end{Description}

                Use '-' flag to reset flags to none. So, \OptArg{-O}{eumx} \OptArg{-O}{t-a} is
                equivalent to \OptArg{-O}{-a}.

				Only \Arg{p} and \Arg{u} have effect in SAM 1.2 format.
\item[\OptArg{--batch-size}{=count}~~\OptArg{-B}{~count}]
				Sets batch size (in count of reads). Too large batch size
                takes too much memory and may lead to excessive paging, too
                small makes scan inefficient. 
\item[\OptArg{--batch-range}{=min\Lbr-max\Rbr}]
                Process only batches in given ordinal range. First batch is 0,
                therefore if one uses \OptArg{-B}{100000} \OptArg{--batch-range}{=2-4} reads from 
                200000 to 499999 will be processed in three batches 100000 in
                each. Convenient for parallel processing.
\end{description}

\subsection{Hashing and scanning options}

\begin{description}
\item[\Opt{--pass0}]
				Hashing and pairing (\Opt{-D}, \Opt{-m}) options which follow this flag will 
                be applied to the first pass.
\item[\Opt{--pass1}]
				Turns on two-pass mode; hashing and pairing (\Opt{-D}, \Opt{-m}) options  
                that follow this flag will be applied to the second pass.
\item[\OptArg{--window-size}{=window\Lbr/word\Rbr}~~\OptArg{-w}{~window\Lbr/word\Rbr}]
                Sets window and word size.
\item[\OptArg{--max-windows}{=count}~~\OptArg{-N}{~count}]
				Set maximal number of consecutive windows to hash.
                Should be 1 (default) if used with \Opt{-k}. Actual number of words
                will be multiplied by stride size. Also alternatives, indels 
                and mismatches will extend this number independently.
\item[\OptArg{--window-start}{=position}~~\OptArg{-r}{~position}]
				Start first hashed window at this position (default is 1)
\item[\OptArg{--window-step}{=dist}~~\OptArg{-f}{~dist}]
				Set step between hashed windows (default is window size plus 
                number of indels allowed in hash)
\item[\OptArg{--max-mismatch}{=mism}~~\OptArg{-n}{~mism}]
				Sets maximal allowed number of mismatches within hashed word.
                Reasonable values are 0 and 1, sometimes 2.
\item[\OptArg{--max-indel}{=0|1|2}~~\OptArg{-e}{~0|1|2}]
				Sets maximal number of indels per window. Allowed values are 
                0, 1, and 2.  Value of 2 allows only single indel of length 
                up to 2 bases.
\item[\OptArg{--max-ins}{=ins}~~\OptArg{-j}{~ins}]
				maximal number of insertions allowed in hash
\item[\OptArg{--max-del}{=del}~~\OptArg{-J}{~del}]
				maximal number of deletions allowed in hash
\item[\OptArg{--max-hash-dist}{=value}~~\OptArg{-E}{~value}]
				Hash only at most with this amount of errors within window 
                (default is max( \Opt{-n}, \Opt{-e}, \Opt{-E} ))
\item[\OptArg{--index-bits}{=bits}~~\OptArg{-H}{~bits}]
				Set number of bits in hash value to be used as an direct index.
                The larger this value the more memory is used.
\item[\OptArg{--stride-size}{=size}~~\OptArg{-S}{~stride}]
				Stride size. Minimum value is 1. Maximum is one base smaller 
                then word size. The larger this value is, the more hash 
                entries are created, but the less hash lookups are performed
                (every "stride's" base of subject). 
\item[\OptArg{--window-skip}{=pos,...}~~\OptArg{-k}{~pos,...}]
				Always skip certain positions of reads when hashing.
                Useful in combination with \OptArg{-n}{0} when it is known that certain
                bases have poor quality for all reads.

                If used with \OptArg{--ambiguity-positions}{=true} (extended option, may
                be changed in next release), just replaces appropriate bases
                with `N's, then this option affects scoring as well.
\item[\OptArg{--indel-pos}{=pos}~~\OptArg{-K}{~pos}]
				Allow indels only at this position of read when hashing
\item[\OptArg{--input-max-amb}{=count}~~\OptArg{-a}{~count}]    
				Maximal number of ambiguous bases in a window allowed for read 
                to be hashed. If \Prog{oligoFAR} fails to find window containing 
                this or lower number of ambiguities for a read, the read is 
                ignored. 
\item[\OptArg{--fasta-max-amb}{=count}~~\OptArg{-A}{~count}]
				Maximal number of ambiguous bases in a window on a subject
                side for the window to be used to perform hash lookup and 
                seed an alignment. 
\item[\OptArg{--phrap-cutoff}{=score}~~\OptArg{-P}{~score}]
				Phrap score for which and below bases are considered as
                ambiguous (used for hashing). The larger it is, the more
                chances to seed lower quality read to where it belongs, but
                also the larger hash table is which decreases performance. 
                Affects 1-channel and 4-channel input. Replaces obsolete 
                \Opt{--solexa-sencitivity} option.
\item[\OptArg{--max-simplicity}{=dust}~~\OptArg{-F}{~dust}]
				Maximal dust score of a window for read to be hashed or for
                the hash lookup to be performed. 
\item[\OptArg{--strands}{=1|2|3}~~\OptArg{-s}{~1|2|3}]
				Strands of database to scan. 1 - positive only, 2 - negative
                only, 3 - both.
\end{description}

\subsection{Alignment options}

\begin{description}
\item[\OptArg{--indel-dropoff}{=value}~~\OptArg{-X}{~value}]
				Longest indel reliably detectable. 0 forbids indels, otherwise 
                for banded Smith-Watermann it controls band width
\item[\OptArg{--extension-dropoff}{=value}~~\OptArg{-x}{~value}]
				maximal score penalty which may be accumulated when extending
                seeded alignment
\item[\OptArg{--band-half-width}{=value}~~\OptArg{-Y}{~value}]
				Band half width for matrix to compute; it makes no sense for X
                to exceed Y. 0 forbids indels.
\item[\OptArg{--longest-ins}{=ins}]
				Longest insertion allowed for reliable detection
\item[\OptArg{--longest-del}{=del}]
				Longest deletion allowed for reliable detection
\item[\OptArg{--max-inserted}{=ins}]
				Maximal number of inserted on extension bases reliably detectable
\item[\OptArg{--max-deleted}{=del}]
				Maximal number of deleted on extension bases reliably detectable
\item[\OptArg{--add-splice}{=pos(\Lbr min:\Rbr max)}
				Add non-penalized splice position and range for alignment
\item[\OptArg{--identity-score}{=score}~~\OptArg{-I}{~score}]
				Set identity score
\item[\OptArg{--mismatch-score}{=score}~~\OptArg{-M}{~score}]
				Set mismatch score
\item[\OptArg{--gap-opening-score}{=score}~~\OptArg{-G}{~score}]
				Gap opening score
\item[\OptArg{--gap-extension-score}{=score}~~\OptArg{-Q}{~score}]
				Gap extention score
\end{description}

\subsection{Filtering and ranking options}

\begin{description}
\item[\OptArg{--min-pctid}{=pct}~~\OptArg{-p}{~pct}]
				Set minimal score for hit to appear in output.  Scores reported are 
                in percent to the best score theoretically possible for the read, so 
                perfect match is always 100.
\item[\OptArg{--top-percent}{=pct}~~\OptArg{-t}{~pct}]
				Set `tie-hit' cutoff. Taking the best hit for the read or read
                pair as 100\%, this is the weakest hit for the read to be
                reported. So, if one has \OptArg{-t}{~90}, and the best hit for some read
                has score 90, report will contain hits with scores between
                90\%*90 = 81 and 90 (provided that \Opt{-p} is 81 or below). This
                option does not affect lines generated by output option \OptArg{-O}{h}.
\item[\OptArg{--top-count}{=cnt}~~\OptArg{-u}{~cnt}]
				Set maximal number of hits per read to be reported. The hits
                reported by ouptut option \OptArg{-O}{h} do not count -- they will be
                reported all.
\item[\OptArg{--pair-distance}{=min\Lbr-max\Rbr}~~\OptArg{-D}{~m\Lbr-M\Rbr}]    
				Set allowed range for distance between component hits for 
                paired read hits. If max is omited it is considered to be equal 
                to min. Length of hits should be included to this length. 
\item[\OptArg{--pair-margin}{=bases}~~\OptArg{-m}{~bases}]
				Set fuzz, or margin for `insert' length for paired reads. The
                actual range of insert lengths allowed is \Lbr a - m, b + m \Rbr,
                where a, b, and m are set with \OptArg{-D}{~a-b} \OptArg{-m}{~m} command line parameters.
\item[\OptArg{--geometry}{=type}~~\OptArg{-R}{~geometry}]
				Sets allowed mutual orientation of the hits in paired read hits. 
                Values allowed (synonyms are separated by `|') are:
	\begin{description}
	\item[p|centripetal|inside|pcr|solexa]
				reads are oriented so that vectors 5'->3' are pointing to each other
	\begin{verbatim}
                ex: >>>>>>>   <<<<<<<
	\end{verbatim}
	\item[f|centrifugal|outside]
				reads are oriented so that vectors 5'->3' are pointing outside
	\begin{verbatim}
                ex: <<<<<<<   >>>>>>>
	\end{verbatim}
	\item[i|incr|incremental|solid]
				reads are on same strand, first preceeds second on this strand
	\begin{verbatim}
                ex: >>>1>>>   >>>2>>>
                or: <<<2<<<   <<<1<<<
	\end{verbatim}
	\item[d|decr|decremental]
				reads are on same strand, first succeeds second on this strand
	\begin{verbatim}
                ex: >>>2>>>   >>>1>>>
                or: <<<1<<<   <<<2<<<
	\end{verbatim}
	\end{description}

                In examples above the pattern >>>1>>> means first component of 
                the paired read on plus strand, <<<2<<< means second component 
                on reverse complement strand; if the digit is not set then the
                component number does not matter for the example.

                Combinations of the values are not allowed.
\end{description}

\subsection{Extended options}
    These options are supposed more for development cycle -- they may choose 
    development versus production algorithm implementations, or set parameters 
    that may not supposed to be exposed in the future. These options should 
    not be used on production and are not guarranteed to be preserved.
\begin{description}
\item[\OptArg{--min-block-length}{=bases}]
                Set minimal length of the subject sequence block to be 
                processed as a whole with same algorithm (depending on 
                presence of ambiguity characters).  Added to tune 
                performance when high density of SNPs is provided. Makes
                no difference with \OptArg{-A}{1}.
\item[\OptArg{--print-statistics}{=yes|no}]
	Turns on or off some output regarding internal \Prog{oligoFAR} statistics
\end{description}

\section{Config file}

	Config file has regular NCBI .ini-file format.  One can specify multiple 
	config files in command line with option \OptArg{-C}{~file}.  Example file
	may be generated with command \Prog{oligofar}~\Opt{--write-config}

	Config file in general may have sections \Lbr oligofar\Rbr, \Lbr oligofar-pass0\Rbr,
	and \Lbr oligofar-pass1\Rbr. Entries in these sections correspond to long options 
	described above (but withour leading double dash "--").  If section \Lbr oligofar-pass1\Rbr
	is present, \Prog{oligoFAR} will run in two-pass mode.

\section{File Formats}\label{secFileFormats}

\subsection{Column-based input file}

    Input file is two to five column whitespace-separated text file. Empty
    lines and lines starting with `#' are ignored (note: they are ignored as
    if not present, so if you use # to comment out a read but provide solexa
    file with -1, or if you use guide file, the corespondense of read ids to 
    the scores or to gided hits will be broken).

    Columns:\\
	\#1 - considered as read-id or read-pair id\\
    \#2 - UIPACna sequence of read \#1\\
    \#3 - UIPACna sequence of read \#2 (optional), or '-' if the read does not have pair\\
    \#4 - quality scores for read \#1 or '-' (optional)\\
    \#5 - quality scores for read \#2 or '-' (optional)

    Quality scores should be represented as ASCII-strings of length equal to 
    appropriate read length, one char per base, with ASCII-value of each char 
    minus 33 representing phrap quality score for the appropriate base.  The 
    number 33 here may be changed with \Opt{-0} (\Opt{--quality-base}) parameter.

    If \OptArg{-q}{1} is used, column 4 is required and should not be '-'; same as column 5 
    if column 3 is not '-' and not empty.

    If \OptArg{-q}{0} is used, columns 3-5 are optional and columns 4, 5 are never used.
    Column 3 is used if present and non-empty and not '-'.

    Mixed (paired and non-paired) input is allowed - rows with columns 3,5 
    containing '-' may interleave rows with all four columns set.

    Example:
	\begin{verbatim}
    rd1  ACAGTAGCGATGATGATGATGATGATWNG  -  ????>????=<?????>>>(876555+!.  -
	\end{verbatim}

    Here in column 4 each char represents a base score for appropriate base in
    column 2, e.g. ? indicates phrap score of 30, > stands for phrap score of 
    29, etc.

\subsection{Input file with di-base colorspace reads (SOLiD technology)}

    Reads may be specified in di-base colorspace encoding.  Option --colorspace=+ 
    should be used, quality scores will be ignored.  SEquence representation should 
    be following: first base is IUPACna, all the rest are digits 0-3 representing 
    dibases:

    0 - AA, CC, GG, or TT\\
    1 - AC, CA, GT, or TG\\
    2 - AG, GA, CT, or TC\\
    3 - AT, TA, CG, or GC

    Example: 

	\begin{verbatim}
    rd1    C02033003022113110030030211    -
	\end{verbatim}
    
    The read above represents sequence CCTTATTTAAGACATGTTTAAATTCAC.

\subsection{Guide file}
    Starting from version 3.99, guide file should be in SAM 1.2 format. 
    
    Also this file should either have \Prog{OligoFAR} version of CIGAR alignments
    (with R for replace, C for changed base, B for overlap), or have field MD:
    set for all reads, because otherwise there is no way to compute compatible 
    alignment score and estimate should the guide alignment be taken or the
    query should be realigned.

    It is essential that order of hits in guide is the same as order of reads in
    input file.

\subsection{Gi list file}
    is just list of integers, one number per line.
    
\subsection{Solexa-style score file}
    Should contain lines of whitespace-separated integer numbers, one line per
    read, 4 numbers per base. Order of reads in the file should correspond to
    order or reads in input file. Input file is required -- it provides read
    IDs. If solexa-style file is set, IUPACna and quality scores from input
    file are ignored.

\subsection{FASTA reads file}
    should be an argument of \Opt{-1} and \Opt{-2} when \OptArg{-q}{0} is used. No spaces are allowed
    after '>' sign. As an extension, colorspace sequence data may be used
    instead of IUPAC.  See note below regarding read identifiers.

\subsection{FASTQ reads file}
    should be an argument of \Opt{-1} and \Opt{-2} when \OptArg{-q}{1} is used.  No spaces are
    allowed after '@' or '+' signs specifying ID lines.  Since now (as of
    version 3.101) oligofar does not support colorspace with quality scores,
    only IUPAC seequence format may be used. See note below regarding read
    identifiers.

\subsection{Read identifiers}
    It is essential that each of the files given by options \Opt{-i}, \Opt{-1}, \Opt{-2}, \Opt{-g}
    contains read information in the same order, exactly one record per read
    (except \Opt{-g} which may have multiple or no records).  Paired reads should
    have the same ID and be in different files (for \Opt{-1}, \Opt{-2}) or columns 
    (for \Opt{-i}); exception is Illumina-style naming: if the reads in \Opt{-1} have ID
    ending by "/1" AND ids in \Opt{-2} are ending by "/2", these last slash and 
    digit are being ignored. 
    
    NB: it is expected that both output and guide SAM do not have "/1"
        and "/2" components in readID.

\subsection{Feature file}
    Is a three-column whitespace-separated file containing 1-based closed
    regions on the reference gequences to scan. 
    Columns are: sequence-id, from, and to.

\section{OUTPUT FORMAT}
    There are two formats avaialble for output: SAM 1.2 and \Prog{oligoFAR}
    proprietary. Default is SAM, although for some purposes proprietary 
    may be more convenient.

    SAM format does not output quality information, sequence data may differ
    from input if quality scoreas are used. Tags AS, XN and XR are generated,
    where XR is rank (see explanation in proprietary output section) and XN is
    number of sequences in output for this rank.

\subsection{Proprietary output format}
    Output file is a 15-column tab-separated file representing different types
    of records (see \Opt{-O} options) in uniform way. 

    Columns:
\begin{enumerate}
%1
\item   Rank of the hit by score (0 - highest), 
        or `*' for unranked hits (see -Oh).
%2  
\item Number of hits of this rank, or "many" if there are more then it
        can be printed by -u option value, or "none" if there are no hits for 
        the query line, or "more" if followed after line "many", or "no_more"
        for terminator lines (see \OptArg{-O}{t}), or "hit" for unranked hits (see \OptArg{-O}{h}),
        or "diff" for lines that present differences of reads to subject 
        (see \OptArg{-O}{d}). See details below.
%3
\item  Query-ID - read or read pair ID.
%4
\item  Subject-ID - one or more IDs of the subject sequence. Which one(s) is 
        not defined here and is implementation dependent.
%5
\item  Mate bitmask: 0 for no hits, 1 if first read matches, 2 if second
        read matches, 3 for paired match.
%6
\item  Total score for the hit (sum or individual read scores)
%7
\item  Position `from' on subject for the read or for the first read of the 
        pair for paired read, or `-' if it is hit for the second read of the 
        pair only.
%8
\item  Position `to' for the read or for the first read of the paired read,
        or `-' if it is hit for the second read of the pair. This position is
        included in the range, and indicates strand (to < from for negatsve 
        strand). 
%9
\item   Score for first read hit, or `-'.
%10
\item Position `from' for the second read, or `-'.
%11
\item Position `to' for the second read (to < from for negative strand), 
        or `-'.
%12
\item Score for second read hit, or `-'.
%13
\item Pair "orientation": '+' if first read is on positive strand, '-'
        if second read is on positive strand.
%14
\item Instantiated as IUPACna image of positive strand of the subject 
        sequence at positions where the first read maps, or '-'.
%15
\item Instantiated as IUPACna image of positive strand of the subject 
        sequence at positions where the second read maps, or '-'.
%17
\item CIGAR alignment of the first read in subject strand, or '-'. 
%18
\item CIGAR alignment of the second read in subject strand, or '-'.
\end{enumerate}

    "Ideal" mapping (having single best) should consider using filter on first 
    column = 0, second column = 1

    Scores for individual reads have maximum of 100, for paired - 200. No
    reads with score below the one set with \Opt{-p} option may appear in output.
    Since filtering by \Opt{-p} is performed before combining individual hits to
    paires, value for \Opt{-p} should not ever exceed 100. 

    CIGAR here uses following letters:\\
        M - match (*)\\
        R - replacement (**)\\
        I - insertion\\
        D - deletion\\
        S - soft masking (dovetail)\\
        C - changed (scored as match although is not the same) (**)\\
        N - splice (not penalized deletion)\\
        B - overlap (subject basea match twice) (**)\\
    Here (*) means that code is changed compared to the extended CIGAR
    (as described in SAM 1.2 standard), and (**) is addition to the CIGAR.

    Following extended CIGAR codes are not produced by oligoFAR:\\
        H - hard masking\\
        P - padding

%Following output flag (a) is currently ignored, 
%but description is preserved for future
%
%    If the flag a for -O is set, for every individual read of hit which score 
%    is below 100 three additional lines will be printed. These lines start
%    with `#' and contain graphical representation of the alignment (in subject
%    coordinates, so that query may appear as reverse complement of the
%    original read):
%
%    # 3'=TTTCCTTTAGA-AGAGCAGATGTTAAACACCCTTTT=5' query[1] has score 68.6
%    #     |||||||||| ||||||| | | ||||||||||||    i:31, m:4, g:1
%    # 5'=ATTCCTTTAGATAGAGCAGTTTTGAAACACCCTTTT=3' subject
%
%    or for di-base colorspace reads:
%
%    # 3'=31200000222133320222333030T=5' query[1]
%    #    ||| |||||||||||||||||||||||    i:26, m:0, g:1
%    # 5'=TAC-TTTTTCTCATATCCTCTATAATT=3' subject
%
%    This format is intended for human review and may be changed in future
%    versions.

\subsubsection{Output record types}
    If a flag `h' was used for -O option, every individual hit with score above
    that was set by -p option will be immediately reported in a line looking
    like
	\begin{verbatim}
    *    hit    ?    ?    ?    ?    ?    ?    ?    ?    ?    ?    ?    ?    ?
	\end{verbatim}

    where ? are set to appropriate values. If the hit bypasses ranking it will
    be also printed as ranked hit after batch is processed. Regular ranked
    hits output may look like:

	\begin{verbatim}
    0  1     rd1  gi|192  1  100  349799  349825  100  - - - +  ACT...  -
    1  3     rd1  gi|195  1  95   799070  799096  95   - - - +  AGT...  -
    1  3     rd1  gi|198  1  95   99070   99096   95   - - - +  ACT...  -
    1  3     rd1  gi|298  1  95   299050  299024  95   - - - -  TTA...  -
    2  many  rd1  gi|978  1  92   267050  267076  92   - - - -  TTA...  -
    2  many  rd1  gi|576  1  92   167050  167076  92   - - - +  ACT...  -
    2  more  rd1  *       *  *    *       *       *    * * * *  *       *
	\end{verbatim}

    Line 1 here means that there is 1 (col 2) hit with best score (col 1) of 
    100 (col 6).

    Lines 2-4 mean that there are also three (col 2) hits with next to best 
    (col 1) score (col 6) of 95.

    Lines 5,6 mean that there are many (col 2) reads of rank 2 in score (if 
    flag x in -O is not used, there will be number 2 in col 2 of these lines).

    Line 7 means that there are more hits of rank 2 and below, which are not
    reported. If flag \Arg{m} in \Opt{-O} is not set, this line will not appear).

    This example suggests that \OptArg{-u}{~6} option was used (that's why no more hits
    are reported).

    If all hits with score above that given by \Opt{-p} are reported, output would
    look like:

	\begin{verbatim}
    0  1       rd1 gi|192  1  100  349799  349825  100  - - - +  ACT...  -
    1  3       rd1 gi|195  1  95   799070  799096  95   - - - +  AGT...  -
    1  3       rd1 gi|198  1  95   99070   99096   95   - - - +  ACT...  -
    1  3       rd1 gi|298  1  95   299050  299024  95   - - - -  TTA...  -
    2  3       rd1 gi|978  1  92   267050  267076  92   - - - -  TTA...  -
    2  3       rd1 gi|576  1  92   167050  167076  92   - - - +  ACT...  -
    2  3       rd1 gi|585  1  92   465010  465036  92   - - - +  ACT...  -
    2  no_more rd1 -       0  0    -       -       0    - - - *  -       -
	\end{verbatim}

    Last line appears only if the flag t of option -O is used.

    If the flag \Arg{d} of option \Opt{-O} is set, every record having score below 100 (or
    below 200 for paired reads) will be followed by one or more records of
    type `diff':

	\begin{verbatim}
    1  diff  rd1  gi|195  1  95  799071  799071  95  -  -  -  +  C=>G  -
	\end{verbatim}

    which means that base C at position 799071 of gi|195 is replaced with G in
    read. Diff lines are always converted to be on positive strand of the
    subject sequence.  Differences longer then one base may be reported,
    IUPACna with ambiguity characters extended with `-' for deletion may be
    used. For paired reads differences for each read are reported in separate
    records, but in appropriate columns (for second read columns 10, 11, 12, 15
    are used instead of 7, 8, 9, 14). Column 13 is always `+'.

    If the flag \Arg{u} of option \Opt{-O} is set, and read has no matches, the line like
    following will appear on output:

	\begin{verbatim}
    0  none  rd2  -  0  0  -  -  0  -  -  0  *  ACG... -
	\end{verbatim}

    where columns 14 and 15 will contain basecalls of the read or reads (if it
    is pair) instead of subject sequence interval.

    Normally all records except `hit' record are clustered by read id (but not
    sorted). If flag \Arg{e} of option \Opt{-O} is set, an empty line is inserted in
    output between blocks of records for different reads to visually separate
    them.

\section{EXIT VALUES}
    0 for success, non-zero for failure.

\LatexManEnd
\end{document}
